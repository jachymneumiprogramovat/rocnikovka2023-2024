\chapter*{Úvod}
% Úvod není číslovaný, třeba ho do obsahu přidat manuálně.
\addcontentsline{toc}{chapter}{Úvod}

% Sem pište úvod.
Míček odrážící se od roviny je v běžném životě velmi častý jev. Ať už jde o
míčové sporty, kde bývá odraz daného míče často nejdůležitější částí celého
sportu. Nebo o srážky kulatých kosmických
těles\footnote{Jedno musí být řádově větší než to druhé aby se dala dopadová plocha
považovat za rovinu}. Pro tuto práci bude důležitější pvní z navrhnutých
využití. 

Motivace tématu této práce pochází ze stolního tenisu. Stolní tenis patří ke
sportům, kde je odraz míčků nejdůležitější část celé hry a možnost ho
odhadnout dává hráči nemalou výhodu. Zajímavým a pro tuto práci hlavním
případem, je odraz míčku, při kterém se míček začne vracet (otočí se rychlost ve
vodorovném směru). Tento typ odrazu můžeme vidět napříč míčovými sporty, můžeme
v nich tedy najít důvod pro studování zpětného odrazu. Popsání podmínek za
kterých dojde ke zpětnému odrazu je hlavním cílem této práce.

Odraz míčku není jediné komplexní chování míčku v míčových sportech. Pro některé
sporty je velmi důležitý pohyb míčku ve vzduchu, nebo méně častěji ve vodě.
Chování míčku v kapalině, i když ho později zanedbáme, bude věnována
zmínka, pro případ že by čtenář toužil prácí rozšířit.


\begin{enumerate}
 \item Případ bez úhlové rotace míčku. (úhel dopadu se rovná úhel odrazu a proč)
 \item Výpis sil působících na míček v průběhu odrazu. (jednoduchý rozbor
  případů - otočení $\omega_2$ atd.
 \item Nastínění předpokladů.
 \begin{enumerate}
  \item Situace těsně před odrazem $ \Rightarrow $ neřeším let míčku
  \item Deformace povrchu nebo míčku (míčku) [zdroj]
 \end{enumerate}
\end{enumerate}
