\chapter*{Úvod}
% Úvod není číslovaný, třeba ho do obsahu přidat manuálně.
\addcontentsline{toc}{chapter}{Úvod}

% Sem pište úvod.
Míček odrážející se od roviny je v běžném životě velmi častý jev. Ať už jde o
míčové sporty, kde bývá odraz daného míče často nejdůležitější částí celého
sportu. Pro tuto práci bude důležitější první z navrhnutých
využití. 

Motivace tématu této práce pochází ze stolního tenisu. Stolní tenis patří ke
sportům, kde je odraz míčků nejdůležitější část celé hry a schopnost ho
odhadnout dává hráči nemalou výhodu. Zajímavým a pro tuto práci hlavním
případem, je odraz míčku, při kterém se míček začne vracet (otočí se rychlost ve
vodorovném směru). Tento typ odrazu můžeme vidět napříč míčovými sporty, můžeme
v nich tedy najít důvod pro studování právě zpětného odrazu. 

Důvodem, již méně \emph{ušlechtilým} za to více upřímným, je pozoruhodnost
zmíněného problému. Možnost kvantifikovat problém nad kterým z čisté zvídavosti
přemýšlím již několik let je velmi přitažlivá. Proto popsání podmínek za
kterých dojde ke zpětnému odrazu je hlavním cílem této práce.
